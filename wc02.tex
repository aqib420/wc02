\documentclass[a4paper]{exam}

\usepackage{geometry}
\usepackage{graphicx}
\usepackage{hyperref}
\usepackage{titling}

\printanswers

\title{Weekly Challenge 02: Polycarp's Permutation!}
\author{$\langle team-name \rangle$}  % <== for grading, replace with your team name, e.g. q1-team-420
\date{Habib University | Spring 2023}

\qformat{{\large\bf \thequestion. \thequestiontitle}\hfill}
\boxedpoints

\begin{document}
\maketitle

\begin{questions}

    \titledquestion{Lost Permutation}
    Once upon a time, in a faraway land, there lived a young mathematician named Polycarp.
    He loved solving problems and was particularly fond of permutations. He would often
    spend hours creating and manipulating different permutations to see what kind of
    patterns he could find.

    One day, while working on a particularly challenging problem, Polycarp got so caught up
    in his work that he didn't realize how much time had passed. When he finally looked up,
    it was dark outside and he realized that he had lost his favorite permutation. Frantic,
    he searched through his papers and notebook, but it was nowhere to be found.Feeling
    defeated, Polycarp decided to try and recreate his permutation by using the numbers
    that he could remember. He had written down the integers $\mathbf{a_1, a_2, \cdots , a_m}$
    but he couldn't remember the rest of the permutation.  However, he is certain that
    sum of the numbers he lost is $\mathbf{S}$.

    Can you find a way to add numbers to the sequence $\mathbf{a_1,a_2,\cdots, a_m}$ that he has
    so that the sum of the added numbers equals $\mathbf{S}$, and the resulting array is a permutation.




    \paragraph{Task}
    \begin{itemize}
        \item Implement the function, \texttt{permute}, in the accompanying file,  \texttt{permutation.py},
              that returns the \textbf{lost permutation} if you can append several elements to the array \textbf{a},
              that their sum equals $\mathbf{S}$ and the result will be a permutation. Return \textbf{NO} otherwise.

        \item A sequence of n numbers is called a permutation if it contains all integers from $1$ to $n$
              exactly once. For example, the sequences $[3,1,4,2], [1]$  and $[2,1]$ are permutations, but $[1,2,1]$,
              $[0,1]$ and $[1,3,4]$ — are not.

    \end{itemize}


    \paragraph{Input}

    \begin{itemize}
        \item The first line of each test set contains two integers $\mathbf{m}$ and $\mathbf{S}$ where $ 1 \leq m \leq 50 $,$1 \leq S \leq 1000 $ where $\mathbf{m}$ is the number of found elements
        and the $\mathbf{S}$ sum of forgotten numbers.
        \item The second line of each test set contains $\mathbf{m}$ \textbf{different integers} $a_1,a_2, \cdots, a_m$ where $1 \leq a_i \leq 50$ are the elements Polycarp managed to find.
    \end{itemize}

    \paragraph{Output}
    \begin{itemize}
        \item Returns the \textbf{lost permutation} (\textit{Permutation can be returned in any order}) if you can append several elements to the array \textbf{a},
        that their sum equals $\mathbf{S}$ and the result will be a permutation. Return \textbf{NO} otherwise.
    \end{itemize}

    \paragraph{Examples}
    \begin{itemize}
        \item For example, when $\mathbf{m}=3 ,\mathbf{S}=13,\mathbf{a}=[3,1,4]$. You can append to $\mathbf{a}$ the numbers $6,2,5,$ the sum of which is $6+2+5=13$. Note that the final permutation which is returned by the function is equal to $[3,1,4,6,2,5]$.
        \item For example, when $\mathbf{m}=3 ,\mathbf{S}=17,\mathbf{a}=[1,3,7]$. You can append to $\mathbf{a}$ the numbers $2,4,5,6$ the sum of which is $2+4+5+6=17$. Note that the final permutation which is returned by the function is equal to $[1,3,7,2,4,5,6]$.
        \item For example, when $\mathbf{m}=3 ,\mathbf{S}=17,\mathbf{a}=[2,4,5]$. You can append to $\mathbf{a}$ the numbers $1,3,6,7$ the sum of which is $1+3+7+6=17$. Note that the final permutation which is returned by the function is equal to $[2,4,5,1,3,6,7]$.
        \item For example, when $\mathbf{m}=1 ,\mathbf{S}=1,\mathbf{a}=[1]$. You cannot append one or more numbers to $\mathbf{a}$ such that their sum equals $1$ and the result is a permutation. Consequently, the function will return \textbf{NO}
        \item For example, when $\mathbf{m}=3 ,\mathbf{S}=17,\mathbf{a}=[2,3,1]$. You can append to $\mathbf{a}$ the numbers $2,4,5,6$ the sum of which is $2+4+5+6=17$. The result is equal to $[2,3,1,2,4,5,6]$ but this not a valid permutation as $2$ is repeated. Consequently, the function will return \textbf{NO}
        
    \end{itemize}
   

\end{questions}

\end{document}